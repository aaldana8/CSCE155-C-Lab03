\documentclass[12pt]{scrartcl}

\setlength{\parindent}{0pt}
\setlength{\parskip}{.25cm}

\usepackage{graphicx}

\usepackage{xcolor}

\definecolor{darkred}{rgb}{0.5,0,0}
\definecolor{darkgreen}{rgb}{0,0.5,0}
\usepackage{hyperref}
\hypersetup{
  letterpaper,
  colorlinks,
  linkcolor=red,
  citecolor=darkgreen,
  menucolor=darkred,
  urlcolor=blue,
  pdfpagemode=none,
  pdftitle={Introduction To Git},
  pdfauthor={Cole Peterson},
  pdfcreator={$ $Id: cv-us.tex,v 1.28 2009/01/01 00:00:00 cbourke Exp $ $},
  pdfsubject={PhD Thesis},
  pdfkeywords={}
}

\definecolor{MyDarkBlue}{rgb}{0,0.08,0.45}
\definecolor{MyDarkRed}{rgb}{0.45,0.08,0}
\definecolor{MyDarkGreen}{rgb}{0.08,0.45,0.08}

\definecolor{mintedBackground}{rgb}{0.95,0.95,0.95}
\definecolor{mintedInlineBackground}{rgb}{.90,.90,1}

%\usepackage{newfloat}
\usepackage[newfloat=true]{minted}
\setminted{mathescape,
               linenos,
               autogobble,
               frame=none,
               framesep=2mm,
               framerule=0.4pt,
               %label=foo,
               xleftmargin=2em,
               xrightmargin=0em,
               startinline=true,  %PHP only, allow it to omit the PHP Tags *** with this option, variables using dollar sign in comments are treated as latex math
               numbersep=10pt, %gap between line numbers and start of line
               style=default, %syntax highlighting style, default is "default"
               			    %gallery: http://help.farbox.com/pygments.html
			    	    %list available: pygmentize -L styles
               bgcolor=mintedBackground} %prevents breaking across pages
               
\setmintedinline{bgcolor={mintedBackground}}
\setminted[text]{bgcolor={mintedBackground},linenos=false,autogobble,xleftmargin=1em}
%\setminted[php]{bgcolor=mintedBackgroundPHP} %startinline=True}
\SetupFloatingEnvironment{listing}{name=Code Sample}
\SetupFloatingEnvironment{listing}{listname=List of Code Samples}

\title{CSCE 155 - C}
\subtitle{Lab 3.0 - Conditionals}
\author{Dr.\ Chris Bourke}
\date{~}

\begin{document}

\maketitle

\section*{Prior to Lab}

Before attending this lab:
\begin{enumerate}
  \item Read and familiarize yourself with this handout.
  \item Review the lecture notes on conditionals, or the following free textbook resources:
	\begin{itemize}
  	  \item \url{http://en.wikibooks.org/wiki/C_Programming/Control}
	  \item \url{http://www.cs.cf.ac.uk/Dave/C/node5.html#SECTION00500000000000000000}
	\end{itemize}
\end{enumerate}

\section*{Peer Programming Pair-Up}

To encourage collaboration and a team environment, labs will be
structured in a \emph{pair programming} setup.  At the start of
each lab, you will be randomly paired up with another student 
(conflicts such as absences will be dealt with by the lab instructor).
One of you will be designated the \emph{driver} and the other
the \emph{navigator}.  

The navigator will be responsible for reading the instructions and
telling the driver what to do next.  The driver will be in charge of the
keyboard and workstation.  Both driver and navigator are responsible
for suggesting fixes and solutions together.  Neither the navigator
nor the driver is ``in charge.''  Beyond your immediate pairing, you
are encouraged to help and interact and with other pairs in the lab.

Each week you should alternate: if you were a driver last week, 
be a navigator next, etc.  Resolve any issues (you were both drivers
last week) within your pair.  Ask the lab instructor to resolve issues
only when you cannot come to a consensus.  

Because of the peer programming setup of labs, it is absolutely 
essential that you complete any pre-lab activities and familiarize
yourself with the handouts prior to coming to lab.  Failure to do
so will negatively impact your ability to collaborate and work with 
others which may mean that you will not be able to complete the
lab.  

\section{Lab Objectives \& Topics}
At the end of this lab you should be familiar with the following
\begin{itemize}
  \item How basic control flow works
  \item When and how to use if, if-then-else, and if-else-if conditional 
  	statements in a program
\end{itemize}

\section{Background}

The default control flow in a typical program is sequential.  That is, 
each statement is executed one after the other.  However, we often 
have to make decisions based on program state or variable values.  
This is known as conditional control flow.  C, like many programming 
languages, provides several control structures to alter the flow of 
control in a program based on the truth-value of some conditional 
statement.

An if-statement can be used to conditionally execute a block of code.  
If the condition in an if-statement evaluates to true the code block is 
executed, otherwise it is not.

\begin{minted}{c}
if(x > 0) {
  printf("x is positive!\n");
  printf("the value of sqrt(x) = %f\n", sqrt(x));
}
\end{minted}

An if-then-else statement can be used to execute one of two mutually 
exclusive code blocks.  If the condition evaluates to true, then the first 
block is executed.  Otherwise, if the condition is false, the second code 
block is executed. 

\begin{minted}{c}
if(x > 0) {
  printf("x is positive!\n");
  printf("the value of sqrt(x) = %f\n", sqrt(x));
} else {
  printf("x is not positive, cannot compute its square root\n");
}
\end{minted}

An if-else-if statement can be used to execute one of several mutually 
exclusive statements.  It also requires more than one conditional 
statement to determine which code block should be executed.   The 
control flow of an if-else-if statement is such that the first condition that 
evaluates to true is executed.  All subsequent statements are skipped.  
If no condition holds true, then the else code block is executed.  
However, much like the difference between an if-statement and an 
if-then-else statement, the final else block is optional.

\begin{minted}{c}
if(x > 0) {
  printf("x is positive!\n");
  printf("the value of sqrt(x) = %f\n", sqrt(x)); 
} else if(x == 0) {
  printf("x is zero, its square root is zero\n");
} else {
  printf("x is not positive, its square root is complex:\n");
  printf("%f\n", sqrt(x*-1.0));
}
\end{minted}

Yet another type of conditional control statement is a switch statement.  
In a switch statement, a single variable's value is used to determine 
which, among several provided cases gets executed.

\begin{minted}{c}
switch(classLevel) {
  case 1:
    printf("Freshman");
    break;
  case 2:
    printf("Sophomore");
    break;
  case 3:
    printf("Junior");
    break;
  case 4:
    printf("Senior");
    break;
  default:
    printf("ERROR: Unknown value!");
    break;
}
\end{minted}

In C, the switch-statement can only be used on integer variables (or 
char variables since they are essentially integers).  Moreover, the 
control flow can easily lead to unexpected results if you forget to 
include the break statement at the end of each case.  

\subsection{Compound Statements}

Many logical statements require the use of logical operators that can 
be used to form more complex, compound statements.  The three 
logical operators that we'll focus on are as follows.

\begin{table}
\centering
\begin{tabular}{l|l|l}
\hline
Operator	& C Syntax & Example \\
\hline\hline
Negation	& \mintinline{c}{!} & \mintinline{c}{!(x > 0)} \\
Logical And & \mintinline{c}{&&} & \mintinline{c}{(x != 0) && (y < 10)} \\ 
Logical Or & \mintinline{c}{||} & \mintinline{c}{(x < 0) || (x >= 10)}
\end{tabular}
\caption{Logical Operators in C}
\label{table:logicalConnectives}
\end{table}

\begin{itemize}
  \item Negation: this is a unary operator that flips the truth value of 
  	the statement or variable that it is applied to so that true becomes 
	false and vice versa.
  \item Logical And: this is a binary operator that is applied to two logical 
	expressions and evaluates to true if and only if both expressions are 
	true
  \item Logical Or: this is a binary operator that is applied to two logical 
	expressions and evaluates to true if at least one (or both) of the 
	expressions are true
\end{itemize}
	
Logical operators are necessary to do more complex statements such as 
checking for ranges of variable values.  For example, to check if a variable 
\mintinline{c}{x} lies in the range $[0, 10]$, one would need to use a logical and operator:

\begin{minted}{c}
if( x >= 0 && x <= 10 ) { 
  //code
}
\end{minted}

\section{Activities}

We have provided partially completed programs for each of the following 
activities.  You will need to clone the Lab 03 project from Github using the URL:
\url{https://github.com/clptrsn/CSCE155-C-Lab03}.  Refer to previous
labs for a step-by-step process.

\subsection{Tax Program}

The federal income tax for a married couple filing jointly for 2012 is determined 
by the rules indicated in Table \ref{table:taxBrackets}.  In addition, the total tax is reduced by 
\$1000 for each child that a couple has.  

\begin{table}
\centering
\begin{tabular}{c|c|c|c}
If the AGI is over- & But not over- & The tax is: & Of the amount over- \\
\hline\hline
\$0 & \$17,000 & 10\% & \$0 \\
\$17,000 & \$69,000 & \$1,700 + 15\% & \$17,000 \\
\$69,000 & \$139,350 & \$9,500 + 25\% & \$69,000 \\
\$139,350 & \$212,300 & \$27,087.50 + 28\% & \$139,350 \\
\$212,300 & \$379,150 & \$47,513.50 + 33\% & \$212,300 \\
\$379,150 & --  & \$102,574.00 + 35\% & \$379,150 \\
\end{tabular}
\caption{2012 Tax Brackets}
\label{table:taxBrackets}
\end{table}

We have provided a partially completed program that reads in a user's 
Adjusted Gross Income (AGI) and prompts whether or not they have 
any children.  If they do it prompts them for how many.  Complete
the program by computing the user's total tax based on the user's AGI 
the number of children they have.

Some example input/output results can be found in Table \ref{table:examples}.
For example, if the user has an adjusted gross income of \$150,000 and has
4 children, then their tax would be calculated as follows.
\begin{itemize}
  \item Their AGI falls in the 4th tax bracket and so would be
    $$\$27,087.50 + 28\% \times (\$150,000 - \$139,350) = \$30,069.50$$
  \item With 4 children, they have a \$4,000 tax credit giving a new total of \$26,069.50
\end{itemize}

\begin{table}
\centering
\begin{tabular}{l|c|l}
AGI & Number of Kids & Taxes \\
\hline\hline
\$4,000 & 1 & \$0 \\
\$20,000 & 0 & \$2,150 \\
\$120,000 & 3 & \$19,250 \\
\$150,000 & 4 & \$26,069.50 \\
\$250,000 & 0 & \$59,954.50 \\
\$500,000 & 5 & \$139,871.50 \\
\end{tabular}
\caption{Several example inputs/outputs.}
\label{table:examples}
\end{table}

Answer the questions on your worksheet.

\subsection{Calculator}

In this activity, you will implement a simple menu-based command line 
calculator.  A partially completed program has been provided to you, 
\mintinline{text}{calculator.c}.  The program prompts the user for two operands 
and one of several different choices for an operation to be performed on 
them.  Your program should process the input and display the result of 
the chosen operation.  Take care with the following possibilities:
\begin{itemize}
  \item For division, you should check if $b = 0$ (division by zero is not defined).  
  	If it is, an appropriate error message should be output instead.
  \item For the logarithm operation, you should use the math library's \mintinline{c}{log} 
	function, which is the natural logarithm (base $e$).  To change to another
	 base, use the formula:
	 	$$\log_a{(b)} = \frac{\ln(b)}{\ln(a)}$$
In addition, you should check that both operands are positive, if not 
then output an appropriate error message.
\end{itemize}


\textbf{Reminder:} when compiling on CSE you may need to direct \mintinline{text}{gcc}
that it needs to \emph{link} the math library by giving it the \mintinline{text}{-lm} 
(\textbf{l}ink \textbf{m}ath) flag; for example:

\mintinline{text}{gcc calculator.c -lm}

Complete the program and answer the questions on your worksheet.

\subsection{Leap Years}

Nearly every 4 years is a \emph{leap year} in the Gregorian calendar.
In a leap year, there are 366 days (adding February 29th) instead of the
usual 365.  Specifically a year $y$ is a leap year if it is divisible
by 4.  However, every year that is divisible by 100 is not a leap year 
unless it is divisible by 400.  Thus, 2000, 2004, 2008 were leaps years
but 2001, 2002, 2003 were not.  1900 was not a leap year; though it was
divisible by 4 it was divisible by 100 but not 400.

We've provided a partially completed program, \mintinline{text}{leapYear.c}
that tests whether or not various years are leap years.  

\begin{enumerate}
  \item Implement a conditional statement inside the 
  \mintinline{c}{isLeapYear()} function to determine if the given
  \mintinline{c}{year} is a leap year or not.  Return true (\mintinline{c}{1})
  if it is, false (\mintinline{c}{0}) if it is not.
  \item Compile and run your program: we've provided 3 
  hard-coded test cases.  Fix any errors in your program until they all 
  \emph{pass}.  
  \item Using the provided code as an example, add at least 3 more
  test cases to your program.  Repeat your compile/run/test until they
  all pass.
\end{enumerate}

Turn your worksheet in to the lab instructor.

\section{Handin/Grader Instructions}

\begin{enumerate}
  \item If you are performing the lab asynchronously, follow these instructions to hand in your code.
  \item Hand in your \mintinline{text}{taxes.c}, \mintinline{text}{calculator.c}, and \mintinline{text}{leapYear.c} source file by pointing your browser to:
  	\url{https://cse-apps.unl.edu/handin} and login with your CSE
	login/password.
  \item Grade yourself by pointing your browser to
  	\url{https://cse.unl.edu/~cse155e/grade/}
  \item Enter your cse login and password, select the appropriate assignment for
  	this lab and click grade me.
  \item You will be displayed with both expected output and your program's output.
	The formatting may differ slightly and that is not important.  As long as your
	program successfully compiles, runs and outputs the same values, it is considered
	correct.
\end{enumerate}


\section{Advanced Activities (Optional)}

\begin{enumerate}
  \item Another conditional operator is the ternary if-then-else operator.  
  	It is often used to choose between two values.  For example:\\	
	\mintinline{c}{int min = ( (a < b) ? a : b );}
	The syntax, \mintinline{c}{X ? Y : Z} is as follows: \mintinline{c}{X} 
	is any conditional statement; if it evaluates to true, then the 
	expression takes on the value \mintinline{c}{Y}; otherwise it takes 
	on the value \mintinline{c}{Z}.  Modify your programs to use this 
	ternary operator where appropriate.
  \item Change the calculator program as follows: add a menu option 
	so that the user has the option to quit; then add a loop so that 
	the program continues to print the menu.  As long as the user 
	performs an operation, it should continue until the user selects 
	the quit option.
\end{enumerate}

\end{document}
